\documentclass{beamer}

\usetheme{default}
\usepackage{graphicx}
\usepackage{booktabs}
\usepackage{xcolor}

% Define custom colors based on styles.css
\definecolor{imdbYellow}{HTML}{F5C518} % IMDb yellow
\definecolor{imdbBlack}{HTML}{000000} % Black background
\definecolor{imdbWhite}{HTML}{FFFFFF} % White text
\definecolor{imdbHoverYellow}{HTML}{E4B10F} % Hover yellow

% Apply the custom colors
\setbeamercolor{background canvas}{bg=imdbBlack}
\setbeamercolor{normal text}{fg=imdbWhite}
\setbeamercolor{frametitle}{bg=imdbYellow, fg=imdbBlack}
\setbeamercolor{title}{fg=imdbYellow}
\setbeamercolor{subtitle}{fg=imdbYellow}
\setbeamercolor{section in toc}{fg=imdbYellow}
\setbeamercolor{item}{fg=imdbYellow}
\setbeamercolor{block title}{bg=imdbYellow, fg=imdbBlack}
\setbeamercolor{block body}{bg=imdbBlack, fg=imdbWhite}

% Set font for better readability
\usepackage{lmodern}
\renewcommand{\familydefault}{\sfdefault}

\title[HistFlix: Sistema de Recomendação Híbrido]{HistFlix: Um Sistema Personalizado para Filmes Históricos e Documentários}
\author{Bruno Baltuilhe, Isaque Pimentel, Kelly Graziely Pena}
\institute{Universidade Presbiteriana Mackenzie}
\date{\today}

\begin{document}

% Slide de Título
\begin{frame}
    \titlepage
\end{frame}

% Slide 1: Introdução
\begin{frame}{Introdução}
    \textbf{HistFlix:} Um sistema de recomendação projetado para sugerir filmes e documentários históricos com base nas preferências e emoções dos usuários.
    \vspace{0.5cm}
    \begin{itemize}
        \item Combina filtragem colaborativa (FC) e filtragem baseada em conteúdo (FBC).
        \item Integra análise de sentimentos para personalizar recomendações.
        \item Objetiva melhorar o aprendizado e o engajamento com conteúdos históricos.
    \end{itemize}
\end{frame}

% Slide 2: Problema
\begin{frame}{Problema}
    \textbf{Desafios em Sistemas de Recomendação:}
    \begin{itemize}
        \item Foco educacional: Necessidade de um sistema voltado para conteúdos históricos e educacionais (ODS da ONU).
        \item Personalização limitada: Recomendações genéricas não capturam emoções dos usuários.
        \item Problema do novo usuário: Falta de dados para novos usuários ou itens.
    \end{itemize}
    \vspace{0.5cm}
    \textbf{Objetivo:} Desenvolver um sistema de recomendação híbrido que combine filtragem colaborativa, baseada em conteúdo e análise de sentimentos.
\end{frame}

% Slide 3: Conjunto de Dados
\begin{frame}{Conjunto de Dados}
    \textbf{Conjunto de Dados MovieLens 1M:}
    \begin{itemize}
        \item 1 milhão de avaliações de 6.000 usuários em 4.000 filmes.
        \item Inclui metadados como gêneros, títulos e anos de lançamento.
    \end{itemize}
    \vspace{0.5cm}
    \textbf{Dados Adicionais:}
    \begin{itemize}
        \item Integrado com a API TMDB para obter:
        \begin{itemize}
            \item Sinopse dos filmes.
            \item Lista de atores.
            \item Avaliações de usuários.
        \end{itemize}
    \end{itemize}
\end{frame}

% Slide 4: Metodologia
\begin{frame}{Metodologia}
    \textbf{Sistema de Recomendação Híbrido:}
    \begin{itemize}
        \item \textbf{FC:} Utiliza SVD (Decomposição em Valores Singulares) para prever avaliações de usuários com base em usuários semelhantes.
        \item \textbf{FBC:} Calcula similaridade usando:
        \begin{itemize}
            \item Gêneros, títulos e anos de lançamento.
            \item Pontuações de sentimentos das avaliações dos usuários.
        \end{itemize}
        \item \textbf{Modelo Híbrido:} Combina pontuações colaborativas e baseadas em conteúdo:
        \[
        \text{Pontuação Híbrida} = \alpha \cdot \text{Pontuação FC} + (1 - \alpha) \cdot \text{Pontuação FBC}
        \]
    \end{itemize}
\end{frame}

% Slide 5: Análise de Sentimentos
\begin{frame}{Próxima Etapa: Análise de Sentimentos}
    \textbf{Integração de Sentimentos:}
    \begin{itemize}
        \item Avaliações de usuários analisadas usando TextBlob.
        \item Pontuações de polaridade variam de -1 (negativo) a +1 (positivo).
        \item Pontuações de sentimentos incluídas no cálculo de similaridade de conteúdo.
    \end{itemize}
    \vspace{0.5cm}
    \textbf{Exemplo:}
    \begin{itemize}
        \item Sentimento positivo: Recomendar gêneros inspiradores (e.g., Comédia, Aventura).
        \item Sentimento negativo: Recomendar gêneros reflexivos (e.g., Drama, Romance).
    \end{itemize}
\end{frame}

% Slide 6: Arquitetura do Sistema
\begin{frame}{Arquitetura do Sistema}
    % \includegraphics[width=\textwidth]{system_architecture.png}
    % \vspace{0.5cm}
    \textbf{Componentes Principais:}
    \begin{itemize}
        \item Pré-processamento e enriquecimento de dados usando API TMDB.
        \item Filtragem colaborativa (SVD).
        \item Filtragem baseada em conteúdo (matriz de similaridade).
        \item Análise de sentimentos (TextBlob).
    \end{itemize}
\end{frame}

% Slide 7: Resultados
\begin{frame}{Resultados}
    \textbf{Métricas de Avaliação:}
    \begin{itemize}
        \item \textbf{RMSE:} 0.87 (quanto menor, melhor).
        \item \textbf{Precisão@10:} 0.82.
        \item \textbf{Revocação@10:} 0.68.
        \item \textbf{F1-Score:} 0.74.
    \end{itemize}
\end{frame}

% Slide 8: Exemplos de Recomendações
\begin{frame}{Exemplos de Recomendações}
    \textbf{Entrada do Usuário:} "Estou feliz e quero assistir algo inspirador."
    \vspace{0.5cm}
    \textbf{Principais Recomendações:}
    \begin{tabular}{ll}
        \toprule
        \textbf{Título} & \textbf{Gêneros} \\
        \midrule
        Toy Story & Animação \\
        Jumanji & Aventura \\
        Sabrina & Comédia \\
        \bottomrule
    \end{tabular}
\end{frame}

% Slide 9: Desafios e Trabalhos Futuros
\begin{frame}{Desafios e Trabalhos Futuros}
    \textbf{Desafios:}
    \begin{itemize}
        \item Problema do novo usuário para novos itens e usuários.
        \item Disponibilidade limitada de avaliações de alta qualidade.
    \end{itemize}
    \vspace{0.5cm}
    \textbf{Trabalhos Futuros:}
    \begin{itemize}
        \item Expandir o conjunto de dados para incluir mais filmes e usuários (e.g., MovieLens 32M).
        \item Melhorar a análise de sentimentos com modelos avançados de PLN (e.g., BERT).
        \item Desenvolver uma interface amigável para recomendações em tempo real.
    \end{itemize}
\end{frame}

% Slide 10: Conclusão
\begin{frame}{Conclusão}
    \textbf{Principais Conclusões:}
    \begin{itemize}
        \item HistFlix combina filtragem colaborativa, baseada em conteúdo e análise de sentimentos.
        \item Oferece recomendações personalizadas e emocionalmente conscientes.
        \item Objetiva melhorar o aprendizado e o engajamento com conteúdos históricos.
    \end{itemize}
    \vspace{0.5cm}
    \textbf{Obrigado!}
    \vspace{0.5cm}
\end{frame}

\end{document}